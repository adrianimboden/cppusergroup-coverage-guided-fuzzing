\section{Intro}

\subsection{Ziel}
\begin{frame}[fragile]
	\frametitle{Ziel heute}

	\begin{itemize}
		\item Praxisbezogen
		\item Selber ausprobieren
	\end{itemize}
	$\rightarrow$ Minimal theoretisch
\end{frame}


\subsection{Was ist Fuzzing}
\begin{frame}[fragile]
	\frametitle{Was ist Fuzzing?}

	Eine Technik, meist automatisiert oder halbautomatisiert, welche:
	\pause\
	\begin{itemize}
		\item ungültigen
		\item unerwartete
		\item oder zufällige Eingabe
	\end{itemize}
	in ein Programm gibt und dann prüft ob die Software:
	\pause\
	\begin{itemize}
		\item Crasht
		\item Asserted
		\item Races beinhaltet
		\item Leaks beinhaltet
		\item \ldots
	\end{itemize}
\end{frame}

\subsection{Anwendungsbeispiel}
\begin{frame}[fragile]
	\frametitle{Beispiel Heartbleed}
	\includesvg[width=.7\textwidth]{pictures/Simplified_Heartbleed_explanation.svg}
\end{frame}

\section{Mein erster eigener Fuzzer}
\subsection{Aufbau}
\begin{frame}[fragile]
	\frametitle{Aufbau}

	my\_fuzzer.cpp:
	\inputminted[fontsize=\scriptsize,linenos]{cpp}{snippets/first_fuzzer.cpp}
	\vspace{5mm}

	Build:
	\begin{minted}[fontsize=\small]{bash}
clang++-8 -g -fsanitize=fuzzer,address my_fuzzer.cpp
  \end{minted}
	\vspace{5mm}

	Run:
	\begin{minted}[fontsize=\small]{bash}
./a.out
  \end{minted}
\end{frame}

\subsection{Optimieren}
\begin{frame}[fragile]
	\frametitle{Optimieren}

	\begin{itemize}
		\item Möglichst kleine Programmteile testen
		\item Möglichst kleine Programmteile testen
	\end{itemize}
\end{frame}

\subsection{Tipps \& Tricks}
\begin{frame}[fragile]
	\frametitle{Tipps \& Tricks}

	\begin{itemize}
		\item Möglichst kleine Programmteile testen
	\end{itemize}
\end{frame}
